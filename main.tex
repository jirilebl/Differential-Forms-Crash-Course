\documentclass[12pt]{article}

\usepackage{enumerate}
\usepackage{amsmath}
\usepackage{amsfonts}
\usepackage{amssymb}
\usepackage{graphicx}
\usepackage{color}
\usepackage{graphics}
\usepackage{eepic}
\usepackage{wrapfig}
\usepackage[a]{esvect}

\usepackage[margin=1in]{geometry}

\usepackage{parskip}

\usepackage[theoremfont]{newpxtext}
\usepackage[varbb]{newpxmath}

\usepackage[T1]{fontenc}


\newcommand{\norm}[1]{\left\lVert {#1} \right\rVert}
\newcommand{\abs}[1]{\left\lvert {#1} \right\rvert}
\newcommand{\snorm}[1]{\lVert {#1} \rVert}
\newcommand{\sabs}[1]{\lvert {#1} \rvert}

\title{Differential Forms Crash Course}
\author{Ji\v{r}\'i Lebl}

\begin{document}

\maketitle

%\section{Differential forms in $\mathbb R^3$}

The classical theorems we learned this semester can be conveniently stated
in a way that gives a vast generalization in one simple statement,
and also allows one to more easily remember/derive the statements of the
theorems, and simplify computations.
We will only scratch the surface (no pun intended) here.

We will mostly worry about 3 dimensions and see how the ideas apply in 2
dimensions.  But the ideas apply in any number of dimensions with almost
no change.

In 3 dimensions there are 4
different kinds of what are called \emph{differential forms}.
There are $0$-forms, $1$-forms, $2$-forms,
$3$-forms.  You have seen $0$-forms and $1$-forms without knowing
about it.  Differential forms are things that are ``integrated'' on the
geometric object of the corresponding dimension (point, path, surface,
region).  In $n$ dimensions there would be $n+1$ different kinds of
differential forms, but let us stick to 3 dimensions for simplicity.

\section*{0-forms}

In the context of differential forms,
functions are called \emph{$0$-forms}.  These $0$-forms are ``integrated''
on points, as points are the $0$-dimensional objects.
That is, functions are evaluated at points.  If $P$ is point, then let
$$
\int_P f = f(P) .
$$
For example,
if $f(x,y,z) = x^2-1+z$ and $P=(1,2,3)$, then
$$
\int_P f = f(1,2,3) = 1^2-1+3 =3 .
$$
Points can have orientation, that is positive or negative.  Above we dealt
with a positively oriented $P$.  If $Q$ is negatively oriented, then 
$$
\int_Q f = -f(Q).
$$
If $Q = (2,1,0)$ is negatively oriented, then
$$
\int_Q f = -f(2,1,0) = -( 2^2-1+0 ) = -3.
$$
We can also add and subtract points.  So suppose that $P=(1,2,3)$ and
$R = (0,0,2)$ are both positively oriented, we write $-P$ as the negatively
oriented $P$ and then we could write $R-P$.  Then
$$
\int_{R-P} f = f(R)-f(P) = f(0,0,2)-f(1,2,3) = 1-3 = -2.
$$
That looks a lot like the ``integral'' of the ``boundary'' of a segment of
a curve that starts at $P$ and ends at $R$, and this is exactly where this
notation will show up.
You then have to be careful
not to do arithmetic on the components of $R-P$, despite what it looks like.
These are points, not vectors, and when points add or subtract, it is in the
sense above.

We haven't really done anything except make up new notation so far, and
it may seem like we're making up nonsense, but the notation will be useful for
stating the fundamental theorem of calculus as the same theorem as 
Green's, Stokes', divergence, etc.

\section*{1-forms}

One-forms are expressions such as
$$
f(x,y,z) \, dx + 
g(x,y,z) \, dy + 
h(x,y,z) \, dz .
$$
For example,
$$
x^2y \, dx + 
3xe^z \, dy + 
(z+y) \, dz .
$$

One-forms are things that are integrated on (oriented) paths, as paths are
one-dimensional.  If $C$ is a path, then we define 
\begin{equation*}
\int_C
f(x,y,z) \, dx + 
g(x,y,z) \, dy + 
h(x,y,z) \, dz 
=
\int_C
\langle f(x,y,z) , g(x,y,z) , h(x,y,z) \rangle \cdot \hat{t} \, ds .
\end{equation*}
And you have seen this expression before.
We use the following formula for the actual computation.
Suppose the path $C$ is parametrized by $t$
for $a \leq t \leq b$.  That is, $x,y,z$ are functions of $t$.  Then
\begin{multline} \label{eq:oneformev}
\int_C
f(x,y,z) \, dx + 
g(x,y,z) \, dy + 
h(x,y,z) \, dz 
\\
=
\int_a^b 
\left( f\bigl(x,y,z\bigr) \frac{d x}{d t}
+
g\bigl(x,y,z\bigr) \frac{d y}{d t}
+ 
h\bigl(x,y,z\bigr) \frac{d z}{d t}
\right) \, dt.
\end{multline}
For example, suppose $C$ is the straight line from $(0,0,0)$ to $(1,2,3)$
parametrized by $x(t) = t$, $y(t) = 2t$, $z(t) = 3t$, for $0 \leq t \leq 1$.
Then
\begin{equation*}
\begin{split}
\int_C
x^2y \, dx + 
3xe^z \, dy + 
(z+y) \, dz 
& =
\int_0^1 
\left( (t)^2 (2t) \bigr) (1)
+
\bigl(3te^{3t} \bigr) (2)
+ 
\bigl(3t+2t\bigr) (3)
\right) \, dt
\\
& = \left[
\frac{t^4}{2}
+
\frac{6t-2}{3} e^{3t}
+
\frac{15t^2}{2}
\right]_{t=0}^1
=
\frac{4}{3} e^3 + \frac{26}{3} .
\end{split}
\end{equation*}

We often give a name to the one-form.  We say
$\omega = f(x,y,z) \, dx + g(x,y,z) \, dy + h(x,y,z) \, dz$.
Then
$$
\int_C \omega = 
\int_C
f(x,y,z) \, dx + 
g(x,y,z) \, dy + 
h(x,y,z) \, dz .
$$

One way that one-forms arise is as derivatives of functions.
For example, let $f$ be a function, then what you called
\emph{total derivative}
in multivariable calculus,
is really the ``$d$ operator'' on $0$-forms giving $1$-forms.  That
is,
$$
df =
\frac{\partial f}{\partial x}\, dx + 
\frac{\partial f}{\partial y}\, dy + 
\frac{\partial f}{\partial z}\, dz .
$$
For example, if $f(x,y,z) = x^2e^yz$, then
$$
df = 
2xe^y z\, dx + 
x^2e^yz\, dy + 
x^2e^y\, dz .
$$
Not every vector field is a gradient vector field, and so
similarly, not every $1$-form is a derivative of a function.

For example, $\omega = -y\, dx + x \, dy$ is not the total derivative
of any function $f$.  If it were, then
\begin{equation*}
\frac{\partial^2 f}{\partial y \partial x} = 
\frac{\partial (-y)}{\partial y} = -1,
\quad \text{but} \quad 
\frac{\partial^2 f}{\partial x \partial y} = 
\frac{\partial (x)}{\partial x} = 1,
\end{equation*}
and that is impossible.

\section*{Boundaries of paths and the fundamental theorem}

If $C$ is a path from point $Q$ to point $P$, then we say that the boundary
of $C$ is $P$ with positive orientation and $Q$ with negative orientation.
This is written as $P-Q$.  We write the boundary as $\partial C = P-Q$.

The upshot of all this is the easy statement of the fundamental theorem
of calculus that will look like all the other statements of the fundamental
theorem.  We can simply write it as
$$
\int_C df = \int_{\partial C} f
$$
Let's interpret this equation.  The left hand side is
$$
\int_C df = \int_C \frac{\partial f}{\partial x}\, dx + 
\frac{\partial f}{\partial y}\, dy + 
\frac{\partial f}{\partial z}\, dz .
$$
While the right hand side, assuming $C$ goes from $Q$ to $P$, is
$$
\int_{\partial C} f = f(P) - f(Q) .
$$
If $f(x,y,z) = 
x^2e^yz$ as above, and $C$ is the path parametrized by
$\gamma(t) = (t,3t,t+1)$ for $0 \leq t \leq 1$, so starting at $(0,0,1)$ and ending
at $(1,3,2)$, then
$$
\int_C df = \int_{\partial C} f = f(1,3,2)-f(0,0,1) = 1^2 e^3 2 - 0^2e^0 1 =
2e^3.
$$

Another example of this use is to compute a path integral by computing the
antiderivative.  For example, suppose $C$ is a straight line from $(0,0,0)$
to $(1,2,3)$, and we want to compute
\begin{equation*}
\int_C y \, dx + x \, dy + 2z \, dz.
\end{equation*}
If we can find an $f$ whose total derivative is the form above, then
we are done.  If $f$ exists then $\frac{\partial f}{\partial x} = y$,
so $f = xy + g(y,z)$ for some function $g$.  Taking the derivative
with respect to $y$ gets us
$\frac{\partial f}{\partial y} = x + \frac{\partial g}{\partial y}$,
so $g$ is independent of $y$.  Taking the derivative with respect
to $z$ we find
$2z = \frac{\partial f}{\partial z} = \frac{\partial g}{\partial z}$,
so $g = z^2$ (plus a constant, but we just need one antiderivative).
So $f = xy+z^2$.  In other words:
\begin{equation*}
\int_C y \, dx + x \, dy + 2z \, dz
=
\int_C df = \int_{\partial C} f
= f(1,2,3) - f(0,0,0) = 1\cdot 2+3^2 = 11.
\end{equation*}

\section*{$2$-forms}

OK, so far we've only seemed to make up notation for things we already know.
For $2$-forms we need to be even more careful with orientation and we
need to keep track of it on the form side of things.  For this we introduce
a new object, the so-called \emph{wedge} or \emph{wedge product}.  It is a
way to put together forms.  In particular, we can
write
$$
dx \wedge dy, \qquad
dy \wedge dz, \qquad
dz \wedge dx .
$$
We define that 
$$
dx \wedge dy = - dy \wedge dx, \qquad
dy \wedge dz = - dz \wedge dy, \qquad
dz \wedge dx = - dx \wedge dz .
$$
Finally, a wedge of something with itself is just zero:
$$
dx \wedge dx = 0, \qquad
dy \wedge dy = 0, \qquad
dz \wedge dz = 0 .
$$
A $2$-form is an expression of the form
$$
\omega = 
f\, dy \wedge dz + 
g\, dz \wedge dx +
h\, dx \wedge dy .
$$
If any other wedges appear, we can use the above rules to convert them to this
form.  For example
$$
x^2 \, dy \wedge dz + 
y \, dx \wedge dz +
z^2\, dx \wedge dx 
=
x^2 \, dy \wedge dz - y \, dz \wedge dx .
$$
We also impose some further algebra rules on this product.  It is really
what we call bilinear.  So if $\omega, \eta$, and $\gamma$ are one-forms, then
$$
(\omega + \eta) \wedge \gamma
=
\omega \wedge \gamma
+
\eta \wedge \gamma ,
$$
and
$$
\omega \wedge (\eta + \gamma)
=
\omega \wedge \eta
+
\omega \wedge \gamma .
$$
Further, if $f$ is a function, then
$$
f \omega \wedge \eta =
f (\omega \wedge \eta) =
\omega \wedge ( f \eta) .
$$
Let's see this on an example:

$
( x^2 y \, dx + z^2 \, dz) \wedge (e^z \, dy + 8 \, dz)
=
x^2 y \, dx \wedge (e^z \, dy + 8 \, dz)
+
z^2 \, dz \wedge (e^z \, dy + 8 \, dz)
\hfill
$

$
\hfill
=
x^2 y e^z \, dx \wedge dy
+
8 x^2 y \, dx \wedge dz
+
z^2 e^z \, dz \wedge dy
+
8 z^2 \, dz \wedge dz
\hfill
$

$
\hfill
=
- z^2 e^z \, dy \wedge dz
-
8 x^2 y \, dz \wedge dx
+
x^2 y e^z \, dx \wedge dy .
$

In general,
\begin{multline*}
(f\, dx + g\, dy + h \, dz) \wedge
(a\, dx + b\, dy + c \, dz)
=
fa\, dx \wedge dx +
fb\, dx \wedge dy +
fc\, dx \wedge dz
\\
+
ga\, dy \wedge dx +
gb\, dy \wedge dy +
gc\, dy \wedge dz 
\\
+
ha\, dz \wedge dx +
hb\, dz \wedge dy +
hc\, dz \wedge dz
\\
=
(gc-hb)\, dy \wedge dz +
(ha-fc)\, dz \wedge dx +
(fb - ga) \, dx \wedge dy .
\end{multline*}
You should recognize the formula for the cross product.  That is,
the result is a 2-form whose coefficients are
$\langle f,g,h \rangle \times
\langle a,b,c \rangle$.
The wedge product
is always the right product in the right context.

OK, now that we know what $2$-forms are, what do we do with them.
First, let's see how to differentiate 1-forms to get 2-forms,
with the $d$ operator.
We want the derivative to be linear, so that in particular
$d(\omega + \eta)= d\omega + d\eta$.  When we have an expression such
as $f \, dx$, we define
$$
d(f \, dx) = df \wedge dx .
$$
Similarly for $dy$ and $dz$.  Let's compute the derivative of
any 1-form:

$
\displaystyle
d ( f \, dx + g \, dy + h \, dz ) = 
df \wedge dx + dg \wedge dy + dh \wedge dz
\hfill
$

$
\displaystyle
\hfill
=
\frac{\partial f}{\partial x} \, dx \wedge dx
+
\frac{\partial f}{\partial y} \, dy \wedge dx
+
\frac{\partial f}{\partial z} \, dz \wedge dx
\hfill
$

$
\displaystyle
\hfill
\hfill
+
\frac{\partial g}{\partial x} \, dx \wedge dy
+
\frac{\partial g}{\partial y} \, dy \wedge dy
+
\frac{\partial g}{\partial z} \, dz \wedge dy
\hfill
$

$
\displaystyle
\hfill
\hfill
+
\frac{\partial h}{\partial x} \, dx \wedge dz
+
\frac{\partial h}{\partial y} \, dy \wedge dz
+
\frac{\partial h}{\partial z} \, dz \wedge dz
\hfill
$

$
\displaystyle
\hfill
=
\left( \frac{\partial h}{\partial y} - \frac{\partial g}{\partial z} \right) \, dy \wedge dz
+
\left( \frac{\partial f}{\partial z} - \frac{\partial h}{\partial x} \right) \, dz \wedge dx
+
\left( \frac{\partial g}{\partial x} - \frac{\partial f}{\partial y} \right)
\, dy \wedge dx .
$

You should recognise the formula for the curl.  That is, if the functions
$f,g,h$ are coefficients of a vector field, then the 
coefficients of the derivative of the one-form
are the coefficients of the curl of the vector field.

For example,
$$
d( x \, dx + y^2 \, dz)
=
2 y \, dy \wedge dz .
$$

The formula $\nabla \times \nabla f = 0$ appears in the fact that
$$
d(df) = 0 .
$$
This is in fact a feature of the $d$ operator, and it is sometimes
written as $d^2 = 0$.

OK, now that we have the derivative, we also want to integrate $2$-forms.
$2$-forms are integrated over surfaces.  Let $S$ be an oriented surface
where $\hat{n}$ is the unit normal that gives the orientation.
We define
$$
\int_S
f\, dy \wedge dz + 
g\, dz \wedge dx +
h\, dx \wedge dy
=
\iint_S \langle f, g, h \rangle \cdot \hat{n} \, dS .
$$
We use only one integral sign for integrals of forms by convention.

A way to compute surface integrals is suggested by the change of
variables formula from multivariable calculus.
Denote
$$
\frac{\partial (x,y)}{\partial (u,v)}
=
\det \left(
\begin{bmatrix}
\frac{\partial x}{\partial u}
&
\frac{\partial x}{\partial v}
\\
\frac{\partial y}{\partial u} 
&
\frac{\partial y}{\partial v}
\end{bmatrix}
\right)
=
\frac{\partial x}{\partial u}
\frac{\partial y}{\partial v}
-
\frac{\partial x}{\partial v}
\frac{\partial y}{\partial u} .
$$
This expression is the determinant of the derivative from the
change of variables formula for 2 dimensional integrals.
This formula is called the Jacobian determinant.
%  Do not worry about what the expression is if you haven't
%seen it.
Let $S$ be parametrized by $(u,v)$ ranging over a domain $D$,
where the ordering $u$ and then $v$
gives the orientation of $S$ via the right hand rule.
So $x$, $y$, and $z$ are functions of $(u,v)$.
Then
\begin{equation*}
\int_S
f\, dy \wedge dz + 
g\, dz \wedge dx +
h\, dx \wedge dy
=
\iint_D
\left(
f 
\frac{\partial (y,z)}{\partial (u,v)}
+
g
\frac{\partial (z,x)}{\partial (u,v)}
+
h
\frac{\partial (x,y)}{\partial (u,v)}
\right)
\,
du\, dv .
\end{equation*}
Compare this to how we computed 1-form integrals above in
equation \eqref{eq:oneformev}, and it will feel very familiar.

For example, let $\omega = x \, dy \wedge dz + y \, dz \wedge dz
+ z \, dx \wedge dy$ be the
$2$-form, and let
$S$ be the surface given by the graph $z=x^2+y^2$ where $x$ and $y$ lie in
the unit square $0 \leq x,y \leq 1$.  We have $x=u$, $y=v$, $z=u^2+v^2$.
The domain $D$ is the unit square $0 \leq u,v \leq 1$.  Then
\begin{multline*}
\int_S \omega =
\int_S
x\, dy \wedge dz + 
y\, dz \wedge dx +
z\, dx \wedge dy
\\
=
\int_0^1 \int_0^1
\left(
u 
(-2u)
+
v
(-2v)
+
(u^2+v^2)
\right)
\,
du\, dv 
\\
=
\int_0^1 \int_0^1
(-u^2-v^2)
\,
du\, dv 
=
\frac{-2}{3} .
\end{multline*}

For another example, suppose $\eta = xyz \, dy \wedge dz$,
and let the surface $S$ be the cylinder of radius 1 around the $z$-axis
for $0 \leq z \leq 1$ oriented with the normal outwards (away from the
$z$-axis).   Let us compute $\int_S \eta$.

First we parametrize $S$.  Let $(u,v)$ map to $(\cos u, \sin u, v)$
for $0 \leq u \leq 2\pi$ and $0 \leq v \leq 1$.  We check that the
right hand rule, curling our fingers around the $u$ direction followed by
the $v$ direction gets us the outward normal.  If it didn't we could just
swap $u$ and $v$.

So
\begin{equation*}
\int_S xyz \, dy \wedge dz
=
\int_0^1
\int_0^{2\pi}
\underbrace{(\cos u)}_{x} \underbrace{(\sin u)}_{y} \underbrace{v}_{z}
\underbrace{( \cos u )}_{\frac{\partial (y,z)}{\partial (u,v)}}
\, du \, dv
= 0 .
\end{equation*}

This is a good way to remember how to integrate parametrized surfaces.

\section*{Stokes' Theorem}

The classical Stokes' Theorem can now be stated.  Let $S$ be an oriented
surface and $\partial S$ be the boundary curve of $S$ oriented according to
the right hand rule as we have for the classical Stokes' Theorem.
Let $\omega$ be a 1-form.  Then Stokes' Theorem in terms of
differential forms is
$$
\int_S d \omega = \int_{\partial S} \omega .
$$
If $\omega = f \, dx + g \, dy + h \, dz$, then
$d \omega$, as we saw above, is really the 2-form whose coefficients are
the components of $\nabla \times \langle f , g, h \rangle$.  So the left 
hand side is
$$
\int_S d \omega = \iint_S 
\nabla \times \langle f , g, h \rangle \cdot \hat{n} \, dS .
$$
The right hand side is the integral 
$$
\int_{\partial S} \omega =
\int_{\partial S} 
\langle f , g, h \rangle \cdot \hat{t} \, ds .
$$
That is,
we have the classical Stokes'.
Notice how the expression
$$
\int_S d \omega = \int_{\partial S} \omega
$$
is now the same for both the Stokes' Theorem and the Fundamental Theorem of
Calculus.  The only difference is that $S$ is a surface or a curve
and 
$\omega$ is a 0-form (function) or a 1-form.

\section*{3-forms}

If we take one more wedge we find that the only forms that survive our
rules, namely that $dx \wedge dx = dy \wedge dy = dz \wedge dz = 0$,
are the ones that look like
$$
f \, dx \wedge dy \wedge dz .
$$
Notice that

$\displaystyle
dx \wedge dy \wedge dz
=
dz \wedge dx \wedge dy
=
dy \wedge dz \wedge dx
\hfill
$

$
\displaystyle
\hfill
=
- 
dy \wedge dx \wedge dz
=
-
dx \wedge dz \wedge dy
=
-
dz \wedge dy \wedge dx .
$

Integrating 3-forms is easy.  Write the 3-form as $f \, dx \wedge dy \wedge
dz$ and then, given a region $R$ in 3-space, we have
$$
\int_R f \, dx \wedge dy \wedge dz
=
\iiint_R f\, dV ,
$$
where $dV$ is the volume measure.  We also put orientation on $R$, and
the above is for positive orientation.  If orientation is not mentioned, we
always mean the positive orientation.  If $R$ would be
oriented negatively, then we define the integral
to be the negative of the integral for positive orientation.
Let us not worry about it, and just do positively oriented
regions in 3-space.

Example:  Let $R$ be the region defined by $-1 < x < 2$, $2 < y < 3$, $0<z<1$.
Then

$
\displaystyle
\int_R x^2 y e^z dx \wedge dy \wedge dz
=
\int_{-1}^2 \int_2^3 \int_0^1 
x^2 y e^z \, dz \, dy \, dx
=
\int_{-1}^2 \int_2^3 
x^2 y (e-1) \, dy \, dx
\hfill
$

$\displaystyle
\hfill
=
\int_{-1}^2
x^2 \left(\frac{3^2}{2} - \frac{2^2}{2}\right) (e-1) \, dx
=
\left( \frac{2^3}{3} - \frac{(-1)^3}{3} \right) \left(\frac{3^2}{2} -
\frac{2^2}{2}\right) (e-1) .
$



Next, how do we differentiate 2-forms to get 3-forms?  We apply essentially
the same formula as before:
$$
d(
f\, dy \wedge dz + 
g\, dz \wedge dx +
h\, dx \wedge dy
)
=
df \wedge dy \wedge dz + 
dg \wedge dz \wedge dx +
dh \wedge dx \wedge dy .
$$
Let us carry this through.  For example, let's start with the first
term:

$
\displaystyle
df \wedge dy \wedge dz 
=
\left(
\frac{\partial f}{\partial x} \, dx +
\frac{\partial f}{\partial y} \, dy +
\frac{\partial f}{\partial z} \, dz 
\right)
\wedge dy \wedge dz 
\hfill
$

$
\displaystyle
\hfill
=
\frac{\partial f}{\partial x} \, dx 
\wedge dy \wedge dz +
\frac{\partial f}{\partial y} \, dy 
\wedge dy \wedge dz +
\frac{\partial f}{\partial z} \, dz 
\wedge dy \wedge dz 
=
\frac{\partial f}{\partial x} \, dx \wedge dy \wedge dz .
$

In the second term, it is only the $\frac{\partial g}{\partial y}$ term to
survive, and in the third term it is only the
$\frac{\partial h}{\partial z}$ term.

All in all we find
that for
$\omega = f\, dy \wedge dz + 
g\, dz \wedge dx +
h\, dx \wedge dy$,
$$
d\omega =
d(
f\, dy \wedge dz + 
g\, dz \wedge dx +
h\, dx \wedge dy
)
=
\left(\frac{\partial f}{\partial x} +
\frac{\partial g}{\partial y} +
\frac{\partial h}{\partial z} \right) \, dx \wedge dy \wedge dz .
$$
And again, notice the expression for the divergence pops up.
We are then not surprised that the Divergence Theorem
$$
\iiint_R \nabla \cdot \langle f,g,h \rangle \, dV = \iint_{\partial R}
\langle f,g,h \rangle \cdot
\hat{n} \, dS,
$$
where $R$ is oriented with the outward unit normal $\hat{n}$
and
$\partial R$ is the boundary of $R$ oriented compatibly,
takes the form
$$
\int_R d \omega = \int_{\partial R} \omega .
$$

The formula
$$
\int_\Omega d \omega = \int_{\partial \Omega} \omega .
$$
is called the \emph{Generalized Stokes' Theorem}.  Here $\omega$ is a
$(k-1)$-form and $\Omega$ is a $k$-dimensional geometric object over which to
integrate.  In 3-space, $\omega$ is a 0, 1, 2, and
$\Omega$ is a path (1-dimensional), a surface
(2-dimensional), or a region (3-dimensional).

\section*{Applying in the plane}

In the plane you can think of everything as if it were in three space but
with no $z$ dependence, so no $dz$.  So there are only 0-forms, 1-forms and 2-forms.
In fact the only $2$-form that appears is the $f dx \wedge dy$ since
the other possibility gets you $dy \wedge dx = - dx \wedge dy$.
The derivative of a one-form is
\begin{equation*}
\begin{split}
d(f \, dx + g \, dy)
& =
df \wedge dx + dg \wedge dy
\\
& =
\left(\frac{\partial f}{\partial x} dx + \frac{\partial f}{\partial y} dy \right)  \wedge dx
+
\left(\frac{\partial g}{\partial x} dx + \frac{\partial g}{\partial y} dy \right)  \wedge dy
\\
& =
\frac{\partial f}{\partial y} dy \wedge dx
+
\frac{\partial g}{\partial x} dx  \wedge dy
\\
& =
\left(\frac{\partial g}{\partial x} -
\frac{\partial f}{\partial y} \right) \,dx  \wedge dy .
\end{split}
\end{equation*}
If $R$ is a region in the plane and $\partial R$ is its boundary, then
the Generalized Stokes' Theorem says:
\begin{equation*}
\begin{split}
\int_{\partial R}
f \, dx + g \, dy
=
\int_R 
d (f \, dx + g \, dy)
=
\int_R
\left(\frac{\partial g}{\partial x} -
\frac{\partial f}{\partial y} \right) \,dx  \wedge dy .
\end{split}
\end{equation*}
And you will recognize Green's Theorem.

\end{document}
